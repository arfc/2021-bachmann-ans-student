\documentclass[12pt, letterpaper]{article}
\usepackage[utf8]{inputenc}
\usepackage{caption} % for table captions
\usepackage{amsmath} % for multi - line equations and piecewises
\usepackage{graphicx}
%\usepackage{textcomp}
\usepackage{xspace}
\usepackage{verbatim} % for block comments
%\usepackage{subfig} % for subfigures
\usepackage{enumitem} % for a) b) c) lists
\newcommand{\Cyclus}{\textsc{Cyclus}\xspace} %
\newcommand{\Cycamore}{\textsc{Cycamore}\xspace} %

\usepackage{tabularx}
\usepackage{color}
\usepackage{setspace}
\definecolor{bg}{rgb}{0.95, 0.95, 0.95}
\newcolumntype{b}{X}
\newcolumntype{f}{ > {\hsize=.15\hsize}X}
\newcolumntype{s}{ > {\hsize=.5\hsize}X}
\newcolumntype{m}{ > {\hsize=.75\hsize}X}
\newcolumntype{r}{ > {\hsize=1.1\hsize}X}
\usepackage{titling}
\usepackage[hang, flushmargin]{footmisc}
\renewcommand *\footnoterule{}
\graphicspath{{images /}}
\usepackage[acronym,toc]{glossaries}
\input{../acros}
\usepackage[margin=1in, voffset=0in]{geometry}
\usepackage{authblk}
\makeatletter
\patchcmd{\@maketitle}{\LARGE \@title}{\fontsize{12}{19.2}\selectfont\@title}{}{}
\makeatother


\renewcommand\Authfont{\fontsize{12}{14.4}\selectfont}
\renewcommand\Affilfont{\fontsize{12}{10.8}\itshape}
\setlength{\parindent}{0.25in}



\title{{\textbf{Enriched Uranium Supply Requirements for the Transition to 
			   Advanced Reactors}}}
\author[1]{Amanda M. Bachmann}
\author[2]{Kathryn Huff }

\affil[1]{\textit{Advanced Reactors and Fuel Cycles, University of Illinois 
at Urbana-Champaign, Department of Nuclear, Plasma, and Radiological 
Engineering, Urbana-Champaign, IL, amandab7@illinois.edu}}
\affil[2]{\textit{Assistant Professor, University of Illinois at 
Urbana-Champaign, Department of Nuclear, Plasma, and Radiological 
Engineering , Urbana-Champaign, IL, 118 Talbot Laboratory, 
kdhuff@illinois.edu
} \vspace{-30pt}}
\date{}


\begin{document}
	\maketitle
	

%Put your content here.
Current nuclear reactors employed in the United States use \gls{LEU} fuel  
enriched to no more than 5\%. New reactor designs, such as the \gls{USNC} 
\gls{MMR}\textsuperscript{TM}, will require \gls{HALEU} fuel enriched 
between 5-20\%. To meet \gls{HALEU} fuel requirements, the U.S. Department 
of Energy is considering recovery and downblending of \gls{HEU} fuel and 
enriching natural uranium to the required levels \cite{griffith_overview_2020}, 
with each of these methods containing their own limitations. Recovery and 
downblending of \gls{HEU} fuel is limited by the existing physical supply 
of \gls{HEU} as well as downblending capacity. 
Enrichment of natural uranium is limited by centrifuge capacity, in 
terms of \gls{SWU}.  

This work aims to quantify the resource requirements of the current 
U.S. reactor fleet and of the transition to different reactors that require 
\gls{HALEU} fuel. Fuel cycle simulations are completed using \Cyclus, an 
agent-based fuel cycle simulator \cite{huff_fundamental_2016}. Transition
scenarios considered include use of the \gls{USNC} \gls{MMR}\textsuperscript{TM} 
and the 
X-energy Xe-100\textsuperscript{TM} reactor, which both require \gls{HALEU} 
fuel. Resource requirements of interest include enriched fuel requirements 
at each enrichment level, \gls{HEU} required to meet \gls{HALEU} demand, 
and natural uranium and \gls{SWU} required to meet \gls{HALEU} demand. 
These metrics will inform the material requirements and provide insight 
into the best method to meet fuel requirements for these transition 
scenarios.  

\bibliographystyle{unsrt}
\bibliography{bibliography}

\end{document}
