%%%%%%%%%%%%%%%%%%%%%%%%%%%%%%%%%%%%%%%%%
% Jacobs Landscape Poster
% LaTeX Template
% Version 1.1 (14/06/14)
%
% Created by:
% Computational Physics and Biophysics Group, Jacobs University
% https://teamwork.jacobs-university.de:8443/confluence/display/CoPandBiG/LaTeX+Poster
% 
% Further modified by:
% Nathaniel Johnston (nathaniel@njohnston.ca)
%
% This template has been downloaded from:
% http://www.LaTeXTemplates.com
%
% License:
% CC BY-NC-SA 3.0 (http://creativecommons.org/licenses/by-nc-sa/3.0/)
%
%%%%%%%%%%%%%%%%%%%%%%%%%%%%%%%%%%%%%%%%%

%----------------------------------------------------------------------------------------
%	PACKAGES AND OTHER DOCUMENT CONFIGURATIONS
%----------------------------------------------------------------------------------------

\documentclass[final]{beamer}

\usepackage[scale=1.0]{beamerposter} % Use the beamerposter package for laying out the poster
\usepackage[acronym,toc]{glossaries}
\input{../acros}
\usetheme{confposter} % Use the confposter theme supplied with this template

\setbeamercolor{block title}{fg=dblue!80,bg=white} % Colors of the block titles
\setbeamercolor{block body}{fg=black,bg=white} % Colors of the body of blocks
\setbeamercolor{block alerted title}{fg=white,bg=dblue!70} % Colors of the highlighted block titles
\setbeamercolor{block alerted body}{fg=black,bg=dblue!10} % Colors of the body of highlighted blocks
% Many more colors are available for use in beamerthemeconfposter.sty

%-----------------------------------------------------------
% Define the column widths and overall poster size
% To set effective sepwid, onecolwid and twocolwid values, first choose how many columns you want and how much separation you want between columns
% In this template, the separation width chosen is 0.024 of the paper width and a 4-column layout
% onecolwid should therefore be (1-(# of columns+1)*sepwid)/# of columns e.g. (1-(4+1)*0.024)/4 = 0.22
% onecolwid should therefore be (1-(# of columns+1)*sepwid)/# of columns e.g. 
% (1-(3+1)*0.025)/3 = 0.3
% Set twocolwid to be (2*onecolwid)+sepwid = 0.464
% Set threecolwid to be (3*onecolwid)+2*sepwid = 0.708

\newlength{\sepwid}
\newlength{\onecolwid}
\newlength{\twocolwid}
\newlength{\threecolwid}
\setlength{\paperwidth}{36in} % A0 width: 46.8in
\setlength{\paperheight}{48in} % A0 height: 33.1in
\setlength{\textwidth}{34in} % A0 width: 46.8in
\setlength{\textheight}{46in} % A0 height: 33.1in
\setlength{\sepwid}{0.025\paperwidth} % Separation width (white space) between columns
\setlength{\onecolwid}{0.3\paperwidth} % Width of one column
\setlength{\twocolwid}{0.625\paperwidth} % Width of two columns
\setlength{\threecolwid}{0.95\paperwidth} % Width of three columns
\setlength{\topmargin}{-0.5in} % Reduce the top margin size
%-----------------------------------------------------------

\usepackage{graphicx}  % Required for including images
\newcommand{\Cyclus}{\textsc{Cyclus}\xspace}%
\usepackage{tabularx}
\newcolumntype{b}{X}
\newcolumntype{s}{>{\hsize=.5\hsize}X}
\newcolumntype{m}{>{\hsize=.75\hsize}X}
\newcolumntype{z}{>{\hsize=.65\hsize}X}

\usepackage{booktabs} % Top and bottom rules for tables
\usepackage{xspace}
\usepackage{amsmath}
\usepackage{exscale}
\usepackage{tikz}
\usetikzlibrary{shapes.geometric, arrows}
\usetikzlibrary{positioning, arrows, decorations, shapes}

\tikzstyle{facility} = [rectangle, rounded corners, minimum width=2cm, minimum height=0.75cm,text centered, draw=black, fill=blue!30]
\tikzstyle{transition} = [rectangle, rounded corners, minimum width=2cm, minimum height=0.75cm,text centered, draw=black, fill=red!30]
\tikzstyle{arrow} = [thick,->,>=stealth]

\setbeamertemplate{bibliography item}[text]

%----------------------------------------------------------------------------------------
%	TITLE SECTION 
%----------------------------------------------------------------------------------------

\title{%
  \texorpdfstring{%
    \makebox[\linewidth]{%
      \makebox[0pt][l]{%
        \raisebox{\dimexpr-\height+\baselineskip}[0pt][0pt]
          {\includegraphics[height=2.5\baselineskip]{UIUC_Logo}}% Left logo
      }\hfill
      \makebox[0pt]{Enriched Uranium Supply Requirements}%
      \hfill\makebox[0pt][r]{%
        \raisebox{\dimexpr-\height+\baselineskip}[0pt][0pt]
          {\includegraphics[height=3.3\baselineskip]{arfc_atom}}% Right logo
      }%
    }%
  }
  % To make a title with two lines, remove the "%" from the following two lines 
  {Enriched Uranium Supply Requirements}
  {for the Transition to Advanced Reactors}
  {\vspace{1cm}}
  } % Poster title

\author{\textbf{Amanda M. Bachmann}, Kathryn D. Huff}
\institute{University of Illinios at Urbana-Champaign, Department of Nuclear, Plasma, and Radiological Engineering, Urbana, IL 61801}
%----------------------------------------------------------------------------------------

\begin{document}

\addtobeamertemplate{block end}{}{\vspace*{2ex}} % White space under blocks
\addtobeamertemplate{block alerted end}{}{\vspace*{2ex}} % White space under highlighted (alert) blocks

\setlength{\belowcaptionskip}{2ex} % White space under figures
\setlength\belowdisplayshortskip{2ex} % White space under equations

\begin{frame}[t] % The whole poster is enclosed in one beamer frame

\begin{columns}[t,totalwidth=\threecolwid] % The whole poster consists of three major columns, the second of which is split into two columns twice - the [t] option aligns each column's content to the top

\begin{column}{0.5\sepwid}\end{column} % Empty spacer column

\begin{column}{\onecolwid} % The first column

%----------------------------------------------------------------------------------------
%	INTRODUCTION
%----------------------------------------------------------------------------------------

\begin{block}{Introduction}

The current fleet of reactors deployed in the United States use
similar forms of \gls{LEU} enriched below 5\% $^{235}$U. Many of the 
new reactor designs will 
use \gls{LEU} fuel enriched between 5-20\% $^{235}$U or \gls{HALEU}. This 
difference in enrichment level requires additional resources to fabricate 
compared to the fuel for current reactors. 

To meet these requirements the U.S. \gls{DOE} has proposed two methods to 
fabricate uranium at \gls{HALEU} levels: recovery and 
downblending of \gls{HEU} or enriching natural uranium  
\cite{griffith_overview_2020}. Each method has limitations, such as physical
supply of \gls{HEU} or \gls{SWU} capacity.

This work investigates the supply of enriched uranium required for three 
different fuel cycle scenarios: the current fleet of U.S. reactors, 
a no-growth transition to the \gls{USNC} \gls{MMR} \textsuperscript{TM}, 
and a no-growth transition to the X-Energy Xe-100 Reactor\textsuperscript{TM}.
This work also considers quantities related to the enriched uranium supply, 
such as number of reactors deployed and \gls{SWU} capacity. 

\end{block}
%----------------------------------------------------------------------------------------
%	OBJECTIVES
%----------------------------------------------------------------------------------------

%This section creates an orange border around a white box
\setbeamercolor{block alerted title}{fg=black,bg=norange} % Change the alert block title colors
\setbeamercolor{block alerted body}{fg=black,bg=white} % Change the alert block body colors
\begin{alertblock}{Objectives}
\begin{itemize}
    \item Quantify the enriched uranium requirements for three fuel cycle 
		  scenarios
	\item Quantify other relevant resources required in the front end of 
		  these three scenarios
\end{itemize}

\end{alertblock}

%----------------------------------------------------------------------------------------
%	Programs
%----------------------------------------------------------------------------------------
\begin{block}{Methodology}
Three different fuel cycle scenarios are simulated:
  \begin{itemize}
    \item current US fleet
    \item transition to \gls{USNC} \gls{MMR}
    \item transition to X-Energy Xe-100
  \end{itemize}
The transition scenarios assume no growth in power demand.


Each scenario is simulated using \Cyclus \cite{huff_fundamental_2016}, an agent 
based fuel cycle simulator. The output of each simulation is parsed for time-dependent 
metrics of interest:
  \begin{itemize}
    \item Number of reactors
    \item Fresh fuel demand of reactors
  \end{itemize}

  \setbeamercolor{block alerted title}{fg=black,bg=norange} % Change the alert block title colors
  \setbeamercolor{block alerted body}{fg=black,bg=white} % Change the alert block body colors
  \begin{alertblock}{Model Assumptions}
  \begin{itemize}
      \item Reactors still operating as of Dec. 2020 operate
        for 60 years after commercial start 
    \item All \gls{LWR}s use the same level enrichment
    \item No new \gls{LWR}s are deployed
    \item Transitions begin at 2025
  \end{itemize}
  
  \end{alertblock}


\end{block}

%----------------------------------------------------------------------------------------

\end{column} % End of the first column

\begin{column}{\sepwid}\end{column} % Empty spacer column


%----------------------------------------------------------------------------------------

\begin{column}{\onecolwid} % The second column
%----------------------------------------------------------------------------------------
%	MODELS
%----------------------------------------------------------------------------------------

\begin{block}{Models}
Figure \ref{fig:fuel_cycle} shows the facilities and flow of materials in 
each of the simulations.  Each non-reactor facility in the 
scenarios was collapsed into a single agent and reactor units are represented 
as separate agents. This allows the simulations to accurately capture the 
deployment and decommissioning of reactor units. 

\begin{figure}
    \centering
    \begin{tikzpicture}[node distance=3.5cm]
        \node (mine) [facility] {Uranium Mine};
        \node (mill) [facility, below of=mine] {Uranium Mill};
        \node (conversion) [facility, below of=mill] {Conversion};
        \node (enrichment) [facility, below of=conversion]{Enrichment};
        \node (fabrication) [facility, below of=enrichment]{Fuel Fabrication};
        \node (reactor) [facility, below of=fabrication]{Reactor};
        \node (wetstorage) [facility, below of=reactor]{Wet Storage};
        \node (drystorage) [facility, below of=wetstorage]{Dry Storage};
        \node (sinkhlw) [facility, below of=drystorage]{HLW Sink};
        \node (sinkllw) [facility, right of=enrichment, xshift=4cm]{LLW Sink};

        \draw [arrow] (mine) -- node[anchor=east]{Natural U} (mill); 
        \draw [arrow] (mill) -- node[anchor=east]{U$_3$O$_8$}(conversion); 
        \draw [arrow] (conversion) -- node[anchor=east]{UF$_6$}(enrichment);
        \draw [arrow] (enrichment) -- node[anchor=east]{Enriched U}(fabrication);
        \draw [arrow] (enrichment) -- node[anchor=south]{Tails}(sinkllw);
        \draw [arrow] (fabrication) -- node[anchor=east]{Fresh UOX}(reactor);
        \draw [arrow] (reactor) -- node[anchor=east]{Spent UOX}(wetstorage);
        \draw [arrow] (wetstorage) -- node[anchor=east]{Cool Spent UOX}(drystorage);
        \draw [arrow] (drystorage) -- node[anchor=east]{Casked Spent UOX}(sinkhlw);

        \end{tikzpicture}
    \caption{Fuel cycle facilities and material flow between facilities.}
    \label{fig:fuel_cycle}
\end{figure}



Deisgn information of the advanced reactors is summarized in Table \ref{tab:reactor_summary}.
	\begin{table}
		\caption{Advanced Reactor design specifications}
		\label{tab:reactor_summary}
		\begin{tabular}{c c c}
			\hline
			Design Criteria & \gls{USNC} \gls{MMR}\textsuperscript{TM} 
      \cite{mitchell_usnc_2020} & X-Energy Xe-100\textsuperscript{TM} 
      \cite{hussain_advances_2018}\\\hline
			Reactor type & Modular HTGR & Modular HTGR \\
			Power (MWth) & 15 & 200 \\
			Enrichment (\% $^{235}U$) & 13 & 15.5 \\
			Cycle Length (years) & 20 & online refuel\\
			Fuel form & TRISO compacts & TRISO pebbles\\
			Reactor Lifetime & 20 years & 60 years \\
			\hline
		\end{tabular}
	\end{table}

\end{block}


%----------------------------------------------------------------------------------------

\end{column} % End of column 2

\begin{column}{\sepwid}\end{column} % Empty spacer column

\begin{column}{\onecolwid} % The third column

\begin{block}{Results}
\textbf{Reactor Deployment}
Figure \ref{fig:rx_deployment} shows the number of reactors as a function of 
time for each scenario.
All of the \gls{LWR}s are decommissioned by 2076
\begin{figure}
  \centering
  \includegraphics[scale=0.65, trim=100 0 50 50, clip]{rx_deployment_2020.png}
  \caption{Number of reactors as a function of time}
  \label{fig:rx_deployment}
\end{figure}


\end{block}

% This section creates an box with an orange border and a white background
\setbeamercolor{block alerted title}{fg=black,bg=norange} % Change the alert block title colors
\setbeamercolor{block alerted body}{fg=black,bg=white} % Change the alert block body colors
\begin{alertblock}{Future Work }
\begin{itemize}
		\item Simulate transitions assuming 1\% growth in demand
		\item Include non-\gls{HALEU} fueled reactors in the simulation
\end{itemize}

\end{alertblock}


%----------------------------------------------------------------------------------------
%	ACKNOWLEDGEMENTS
%----------------------------------------------------------------------------------------

\setbeamercolor{block title}{fg=norange,bg=white} % Change the block title color

\begin{block}{Acknowledgements}
	
	This material is based upon work supported under an Integrated University 
Program Graduate Fellowship. Any opinions, findings, conclusions, or 
recommendations expressed in this publication are those of the author(s) 
and do not necessarily reflect the views of the Department of Energy Office 
of Nuclear Energy.

Prof. Huff is supported by the Nuclear Regulatory Commission Faculty
Development Program (award NRC-HQ-84-14-G-0054 Program B), the Blue Waters
sustained-petascale computing project supported by the National Science
Foundation (awards OCI-0725070 and ACI-1238993) and the state of Illinois, the
DOE ARPA-E MEITNER Program (award DE-AR0000983), and the DOE H2@Scale Program
(Award Number: DE-EE0008832)
	
\end{block}

%----------------------------------------------------------------------------------------
%	CONTACT INFORMATION
%----------------------------------------------------------------------------------------

\setbeamercolor{block alerted title}{fg=white,bg=dblue} % Change the alert block title colors
\setbeamercolor{block alerted body}{fg=black,bg=white} % Change the alert block body colors

\begin{alertblock}{Contact Information}
	\setbeamercolor{block title}{fg=norange,bg=white} % Change the block title color
	\begin{itemize}
		\item Email: \href{mailto:amandab7@illinois.edu}{amadnab7@illinois.edu}
		\item Phone: \href{tel:12173003132}{+1 217-300-3132}
	\end{itemize}
% These specific elements are optional, but you should have some method for people to contact you.	
	
\end{alertblock}

\begin{block}{References}

	{\footnotesize\bibliographystyle{abbrv} 
	\bibliography{poster}}
\end{block}


%----------------------------------------------------------------------------------------



\end{column} % End of the third column

\end{columns} % End of all the columns in the poster

\end{frame} % End of the enclosing frame

\end{document}
\begin{column}{\sepwid}\end{column} % Empty spacer column
